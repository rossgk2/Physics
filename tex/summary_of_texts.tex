
\textbf{Tong}
\begin{itemize}
    \item Gives the definition $F(x, x’) = p’$ extremely tersely, not mentioning that the definition is what it is because momentum is conserved. (He later relies on the same definition parsing argument I did to prove conservation of momentum.)
    \item The perspective of force as being the formalization of the determinancy axiom is emphasized.  
    \item Doesn’t motivate $L = T - U$
\end{itemize}

\textbf{Taylor}
\begin{itemize}
    \item Extremely typical and lazy intuition in terms of “pushes” is given for$ F = ma$. No justification provided for why a “push” corresponds to a change in velocity and not a change in position. And even makes the horrible mistake of claiming that in classical mechanics $F = ma$ is “completely equivalent” to $F = dp/dt$.
    \item Falls prey to my old mistake of proving conservation of momentum via definition parsing.
    \item Gives interesting discussion on Fermat’s principle to motivate the principle of stationary action. But doesn’t motivate $L = T - U$.
\end{itemize}

\textbf{Arnold}
\begin{itemize}
    \item Emphasizes the perspective that force is the result of the determinancy axiom.
    \item Makes use of the axiom that the motion of a system is determined from the initial positions and velocities of particles in the system.
    \item Doesn’t motivate $L = T - U$.
    \item Shows that E-L for constant mass is equivalent to d’Alembert’s principle, but only after much discussion on the action principle and E-L.
\end{itemize}